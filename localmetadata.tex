\title{Formulaic language}
\subtitle{{T}heories and methods}
\BackBody{The notion of formulaicity has received increasing attention in disciplines and areas as diverse as linguistics, literary studies, art theory and art history. In recent years, linguistic studies of formulaicity have been flourishing and the very notion of formulaicity has been approached from various methodological and theoretical perspectives and with various purposes in mind. The linguistic approach to formulaicity is still in a state of rapid development and the objective of the current volume is to present the current explorations in the field. Papers collected in the volume make numerous suggestions for further development of the field and they are arranged into three complementary parts. The first part, with three chapters, presents new theoretical and methodological insights as well as their practical application in the development of custom-designed software tools for identification and exploration of formulaic language in texts. Two papers in the second part explore formulaic language in the context of language learning. Finally, the third part, with three chapters, showcases descriptive research on formulaic language conducted primarily from corpus linguistic and translation studies perspectives. The volume will be of interest to anyone involved in the study of formulaic language either from a theoretical or a practical perspective.
}
\typesetter{Sebastian Nordhoff, Felix Kopecky}
% \proofreader{}
\author{Aleksandar Trklja and Łukasz Grabowski}

% \BookDOI{}
\renewcommand{\lsISBNdigital}{000-0-000000-00-0}
\renewcommand{\lsISBNhardcover}{000-0-000000-00-0}
\renewcommand{\lsSeries}{pmwe}
\renewcommand{\lsSeriesNumber}{5}
\renewcommand{\lsID}{304}

\renewcommand{\lsCoverTitleFont}[1]{\sffamily\addfontfeatures{Scale=MatchUppercase}\fontsize{46pt}{15.25mm}\selectfont #1}
